\documentclass[12pt,halfline,a4paper]{ouparticle}

\usepackage{lmodern}
\usepackage[T1]{fontenc}
\usepackage{textcomp}

\usepackage{graphicx}
\graphicspath{ {../grafici/} }

\begin{document}

\title{Report\break {Sistemi di Calcolo Parallelo e Applicazione}}

\author{%
\name{Valerio Cristofori}
%\address{Institute or Organization, Department, City, State,\\
%Zip Code, Country}
%\email{e-mail address}
%\and
%\name{Second Author}
%\address{Institute or Organization, Department, City, State,\\
%Zip Code, Country}
%\email{e-mail address}
}

\abstract{In questo documento vengono argomentate le scelte effettuate 
per realizzare un nucleo di calcolo per il prodotto tra una matrice sparsa e un vettore.\\
Vengono poi presentati e analizzati i risultati del calcolo.} 

\date{AA 2021/2022}

\maketitle

\section{Introduzione}
\label{sec1}

L'obiettivo è realizzare un nucleo di calcolo parallelo, con 2 tipi di parallelizzazione.\break
La prima è basata su un'implementazione del concetto di multithreading su sistemi a memoria condivisa: OpenMP.
La seconda è CUDA, ovvero un'API che permette l'uso di GPU per il calcolo parallelo (general-purpose computing su GPU).
Ogni tipo di parallelizzazione viene poi confrontata con l'implementazione di un nucleo di calcolo seriale.\\
La fase di calcolo è preceduta da una fase di preprocessamento e memorizzazione dal formato \verb+MatrixMarket+ ai formati di memorizzazione per matrici sparse: \verb+CSR+ e \verb+ELLPACK+.\\
Le matrici usate per il test sono alcune matrici della “Suite Sparse Matrix Collection”, disponibili all'indirizzo \verb+https://sparse.tamu.edu/+.


\section{Preprocessamento}
\label{sec2}
\subsection{Caricamento e Formattazione}
\label{sec2.1}

Sono state usate le funzioni di libreria fornite all'indirizzo \verb+http://math.nist.gov/MatrixMarket/+ per caricare le matrici dal formato \verb+MatrixMarket+ dove vengono specificati il numero di righe \textit{M}, numero di colonne \textit{N}, e il numero di non zeri \textit{NZ}.\\
Da notare che per le matrici simmetriche viene memorizzato su file solo un triangolo (superiore/inferiore), di conseguenza, in memoria, vengono ricostruiti gli array di indici di riga, colonna e valori dell'intera matrice, e aggiornato il numero dei \textit{NZ}.
È stato aggiunto un controllo sulle matrici memorizzate come \textit{"pattern"}, popolando il vettore relativo ai valori nonzero di coefficienti di valore 1.0.\\\\
Sono stati formati in memoria i seguenti vettori:\\
\indent\textbf{I(1:NZ)}: Vettori degli indici di riga dei coefficienti nonzero\\
\indent\textbf{J(1:NZ)}: Vettori degli indici di colonna dei coefficienti nonzero\\
\indent\textbf{val(1:NZ)}: Vettori dei valori dei coefficienti nonzero\\
Viene poi costruito un vettore di indici \textbf{idxs} della matrice, in modo tale che l'i-esima posizione sia relativa a \textit{I[i] * N + J[i]}. Vengono poi ordinati i vettori per indici di riga (si presuppone che i dati memorizzati su file siano ordinati per colonna). In questo modo il \textbf{idxs} risultante rappresenta gli indici dei valori nonzero della matrice, riga per riga.\\


\subsection{Formattazione CSR}
\label{sec2.2}

\textit{Compressed Storage by Rows}. In questo caso la matrice viene formattata nel seguente modo:\\
\indent\textbf{IRP(1:M+1)}: Vettore dei puntatori all’inizio di ciascuna riga\\
\indent\textbf{JA(1:NZ)}: Vettore degli indici di colonna\\
\indent\textbf{AS(1:NZ)}: Vettore dei coefficienti\\
Viene costruito il vettore \textit{IRP} partendo dal vettore \textit{idxs}. È stato tenuto conto della possibile presenza di matrici con righe senza valori nonzero che portavano ad un calcolo errato (con valori di \textit{IRP} a -1, come default): in questo caso le righe sono state scartate, ed è stato aggiornato il valore del numero di righe \textit{M}.

\subsection{Formattazione ELLPACK}
\label{sec2.3}

In questo caso la matrice viene formattata nel seguente modo:\\
\textbf{JA(1:M,1:maxnz)}: array 2D di indici di colonna\\
\textbf{AS(1:M,1:maxnz)}: array 2D di coefficienti\\
La formattazione ELLPACK differisce dalla tipologia di parallelizzazione:
\subsubsection{Formattazione ELLPACK per OpenMP}
\label{sec2.3.1}
Nel caso di formattazione ELLPACK per parallelizzazione con OpenMP il valore di \textit{maxnz} è fisso e pari al massimo valore di nonzeri per riga, su tutte le righe. Una volta trovato il valore \textit{maxnz} si popolano i vettori \textit{JA} e \textit{AS}.\\
Da notare che per ogni riga che conta un numero di nonzeri minore di \textit{maxnz} i rispettivi coefficienti sono nulli per \textit{AS} e sono uguali all'ultimo indice valido incontrato lungo la riga per \textit{JA}.\\


\subsubsection{Formattazione ELLPACK per CUDA}
\label{sec2.3.2}
Per quanto riguarda CUDA invece oltre a trovare il valore \textit{maxnz} e inizializzare i vettori \textit{JA} e \textit{AS} viene costruito anche il vettore \textbf{MAXNZ(1:M)}, vettore di numeri di nonzeri per ogni riga.\\
Successivamente viene eseguita la trasposizione degli array 2D \textit{JA} e \textit{AS}. Rendendo trasposti i dati i coefficienti di ciascuna colonna risultano adiacenti in memoria essendo \textit{C} un linguaggio con memorizzazione per righe.\\


\section{Nucleo di calcolo}
\label{sec3}

\subsection{Calcolo Seriale}
\label{sec3.1}


\begin{center}
\begin{tabular}{|| c | c | c ||} 
\hline
 & \multicolumn{2}{c||}{Speedup CPU}\\
\hline
Matrice&CSR&ELLPACK\\
\hline
adder\_dcop\_32 & 0.21 & 0.01 \\
\hline
af\_1\_k101 & 9.68 & 8.80 \\
\hline
af23560 & 2.92 & 2.77 \\
\hline
amazon0302 & 5.81 & 5.83 \\
\hline
bcsstk17 & 1.71 & 1.07 \\
\hline
cage4 & 0.00 & 0.00 \\
\hline
cant & 4.74 & 5.92 \\
\hline
cavity10 & 0.53 & 0.47 \\
\hline
cop20k\_A & 7.09 & 2.66 \\
\hline
Cube\_Coup\_dt0 & 10.14 & 7.99 \\
\hline
dc1 & 2.31 & nd \\
\hline
FEM\_3D\_thermal1 & 2.28 & 2.39 \\
\hline
lung2 & 3.26 & 2.43 \\
\hline
mac\_econ\_fwd500 & 3.69 & 1.29 \\
\hline
mcfe & 0.23 & 0.18 \\
\hline
mhd4800a & 0.73 & 0.65 \\
\hline
mhda416 & 0.12 & 0.04 \\
\hline
ML\_Laplace & 9.96 & 9.01 \\
\hline
nlpkkt80 & 9.74 & 8.75 \\
\hline
olafu & 3.39 & 3.20 \\
\hline
olm1000 & 0.05 & 0.06 \\
\hline
PR02R & 8.47 & 4.58 \\
\hline
raefsky2 & 1.48 & 1.23 \\
\hline
rdist2 & 0.48 & 0.23 \\
\hline
roadNet-PA & 5.93 & 3.20 \\
\hline
thermal1 & 2.31 & 3.44 \\
\hline
thermal2 & 3.93 & 5.57 \\
\hline
thermomech\_TK & 4.53 & 4.24 \\
\hline
webbase-1M & 6.09 & nd \\
\hline
west2021 & 0.10 & 0.08 \\
\hline
\end{tabular}
\end{center}



\begin{center}
\begin{tabular}{|| c | c | c ||} 
\hline
 & \multicolumn{2}{c||}{Speedup GPU}\\
\hline
Matrice&CSR&ELLPACK\\
\hline
adder\_dcop\_32 & 1.78 & 0.78 \\
\hline
af\_1\_k101 & 66.32 & 110.73 \\
\hline
af23560 & 35.67 & 72.48 \\
\hline
amazon0302 & 21.79 & 72.35 \\
\hline
bcsstk17 & 26.97 & 46.71 \\
\hline
cage4 & 0.00 & 0.00 \\
\hline
cant & 75.58 & 81.33 \\
\hline
cavity10 & 6.81 & 17.19 \\
\hline
cop20k\_A & 50.65 & 69.95 \\
\hline
Cube\_Coup\_dt0 & 110.93 & 119.84 \\
\hline
dc1 & 8.32 & nd \\
\hline
FEM\_3D\_thermal1 & 28.77 & 61.26 \\
\hline
lung2 & 9.68 & 24.02 \\
\hline
mac\_econ\_fwd500 & 16.89 & 32.47 \\
\hline
mcfe & 2.41 & 4.21 \\
\hline
mhd4800a & 10.38 & 23.35 \\
\hline
mhda416 & 1.25 & 2.27 \\
\hline
ML\_Laplace & 113.11 & 115.12 \\
\hline
nlpkkt80 & 69.91 & 109.88 \\
\hline
olafu & 53.04 & 79.07 \\
\hline
olm1000 & 0.52 & 1.28 \\
\hline
PR02R & 81.70 & 65.85 \\
\hline
raefsky2 & 18.29 & 39.91 \\
\hline
rdist2 & 4.83 & 11.35 \\
\hline
roadNet-PA & 14.93 & 106.62 \\
\hline
thermal1 & 16.81 & 39.09 \\
\hline
thermal2 & 30.21 & 111.85 \\
\hline
thermomech\_TK & 21.35 & 48.29 \\
\hline
webbase-1M & 13.11 & nd \\
\hline
west2021 & 0.70 & 2.00 \\
\hline
\end{tabular}
\end{center}


\begin{center}
\begin{tabular}{|| c | c | c ||} 
\hline
 & \multicolumn{2}{c||}{GFlops CPU}\\
\hline
Matrice&CSR&ELLPACK\\
\hline
adder\_dcop\_32 & 0.04 & 0.00 \\
\hline
af\_1\_k101 & 4.67 & 4.24 \\
\hline
af23560 & 1.10 & 1.10 \\
\hline
amazon0302 & 1.02 & 1.30 \\
\hline
bcsstk17 & 0.85 & 0.49 \\
\hline
cage4 & 0.00 & 0.00 \\
\hline
cant & 2.31 & 2.44 \\
\hline
cavity10 & 0.21 & 0.17 \\
\hline
cop20k\_A & 2.26 & 0.86 \\
\hline
Cube\_Coup\_dt0 & 5.04 & 3.99 \\
\hline
dc1 & 0.86 & nd \\
\hline
FEM\_3D\_thermal1 & 0.92 & 0.94 \\
\hline
lung2 & 1.07 & 0.49 \\
\hline
mac\_econ\_fwd500 & 1.48 & 0.46 \\
\hline
mcfe & 0.08 & 0.04 \\
\hline
mhd4800a & 0.25 & 0.21 \\
\hline
mhda416 & 0.03 & 0.03 \\
\hline
ML\_Laplace & 4.83 & 4.36 \\
\hline
nlpkkt80 & 4.67 & 4.24 \\
\hline
olafu & 1.65 & 1.42 \\
\hline
olm1000 & 0.02 & 0.02 \\
\hline
PR02R & 4.14 & 2.25 \\
\hline
raefsky2 & 0.46 & 0.39 \\
\hline
rdist2 & 0.15 & 0.07 \\
\hline
roadNet-PA & 1.38 & 0.75 \\
\hline
thermal1 & 1.17 & 1.02 \\
\hline
thermal2 & 0.93 & 1.60 \\
\hline
thermomech\_TK & 1.13 & 0.99 \\
\hline
webbase-1M & 1.66 & nd \\
\hline
west2021 & 0.03 & 0.03 \\
\hline
\end{tabular}
\end{center}


\begin{center}
\begin{tabular}{|| c | c | c ||} 
\hline
 & \multicolumn{2}{c||}{GFlops GPU}\\
\hline
Matrice&CSR&ELLPACK\\
\hline
adder\_dcop\_32 & 0.44 & 0.19 \\
\hline
af\_1\_k101 & 31.85 & 53.18 \\
\hline
af23560 & 15.37 & 31.24 \\
\hline
amazon0302 & 4.86 & 16.14 \\
\hline
bcsstk17 & 12.07 & 20.91 \\
\hline
cage4 & 0.01 & 0.01 \\
\hline
cant & 35.46 & 38.17 \\
\hline
cavity10 & 2.88 & 7.27 \\
\hline
cop20k\_A & 18.10 & 24.99 \\
\hline
Cube\_Coup\_dt0 & 53.89 & 58.22 \\
\hline
dc1 & 3.30 & nd\\
\hline
FEM\_3D\_thermal1 & 13.05 & 27.79 \\
\hline
lung2 & 3.75 & 9.29 \\
\hline
mac\_econ\_fwd500 & 6.40 & 12.30 \\
\hline
mcfe & 1.00 & 1.74 \\
\hline
mhd4800a & 4.54 & 10.23 \\
\hline
mhda416 & 0.43 & 0.78 \\
\hline
ML\_Laplace & 53.51 & 54.45 \\
\hline
nlpkkt80 & 33.36 & 52.43 \\
\hline
olafu & 24.76 & 36.91 \\
\hline
olm1000 & 0.18 & 0.44 \\
\hline
PR02R & 38.61 & 31.12 \\
\hline
raefsky2 & 8.17 & 17.83 \\
\hline
rdist2 & 2.11 & 4.95 \\
\hline
roadNet-PA & 3.60 & 25.70 \\
\hline
thermal1 & 5.55 & 12.91 \\
\hline
thermal2 & 8.91 & 33.00 \\
\hline
thermomech\_TK & 5.67 & 12.82 \\
\hline
webbase-1M & 3.89 & nd \\
\hline
west2021 & 0.23 & 0.67 \\
\hline
\end{tabular}
\end{center}

\subsection{Calcolo con OpenMP}
\label{sec3.2}

Before you type anything that actually appears in the paper, you need to
include a \verb+\documentclass{ouparticle}+ command at the very beginning,
and then the two commands that have to be part of any \LaTeX\ document,
\verb+\begin{document}+ at the start and \verb+\end{document}+ at the
end of your paper.


\subsection{Calcolo con CUDA}
\label{sec3.3}

The main structure of your paper is as follows:


\section{Performance}
\label{sec4}

By default, all of the options within \verb+article.cls+ are available
with this class file. This class file provides the following additional options.


\subsection{Front matter}
\label{sec4.1}

The title of the manuscript is simply specified by using the \verb+\title{text}+ command in
the same manner as in this sample. Author's information consists of the name of the author
and the corresponding institutions with addresses, as given in this example. Include an
electronic mail address if available, inserting it into the \verb+\email{text}+ commands.
You may follow the same coding if there are more than one author; separate authors with
\verb+\and+. Please identify the corresponding author with his/her electronic
mail address by \verb+\thanks{text}+. An abstract for your paper is specified by using
\verb+\abstract{text}+. A \verb+\keywords{text}+ macro may also be used to indicate keywords for the
article. Use \verb+\maketitle+ after the abstract and keywords to make the header of your article.

\subsection{Sections and subsections}
\label{sec4.2}

To begin a new section, give the heading of that section in the \verb+\section{text}+ command.
A section number is supplied automatically. Use the starred form (\verb+\section*{text}+) of the
command to suppress the automatic numbering. If you want to be able to make reference to that section,
then you need to \texttt{label} it (see Section \ref{sec3.14}). You can have sections up to
five levels. The sectioning commands are \verb|\section|, \verb|\subsection|, \verb|\subsubsection|,
\verb|\paragraph| and \verb|\subparagraph|.

\section{Conclusioni}
\label{sec5}

The ends of words and sentences are marked by spaces. It does not matter how many
spaces you type. The end of a line counts as a space. One
or more blank lines denote the end of a paragraph.


\subsection{Appendix}
\label{sec6.1}

The \verb+\appendix+ command signals that all following sections are
appendices, and therefore the headings after \verb+\appendix+ will be set
as appendix headings. For a single appendix, use \verb+\appendix*+ followed by the \verb+\section{text}+
command to suppress the appendix letter in the section heading.




\section{References}
\label{sec6}

The reference entries can be \LaTeX\ typed bibliographies or generated through a BIB\TeX\ database.
BIB\TeX\ is an adjunct to \LaTeX\ that aids in the preparation of bibliographies. BIB\TeX\
allows authors to build up a database or collection of bibliography entries that may be used for many
manuscripts. They also save us the trouble of having to specify formatting. More details can be found
in the \textit{BIB\TeX\ Guide}. For \LaTeX\ reference entries use the
\verb+\begin{thebibliography}....\end{thebibliography}+ environment (see below) to make references in your paper.
We have provided the class file option to distinguish two styles of references. Those options are \verb+numbib+ and \verb+nonumbib+.
You can select one of these options with the \verb+\documentclass+ command. By default the class file will take the
\verb+numbib+ option. The following is an example of \LaTeX\ bibliography.

\begin{verbatim}
\begin{thebibliography}{0}
\bibitem{bib1}
Goossens, M., F. Mittelbach, and A. Samarin: {\em The {\LaTeX} Companion}.
Addison-Wesley, Reading, MA, USA, 1994.
\bibitem{bib2}
Knuth, D.E: {\em The {\TeX}book}. Addison-Wesley, Reading, MA, USA, 1984.
\bibitem{bib3}
Lamport, L.: {\em {\LaTeX} -- A Document Preparation System -- User's
Guide and Reference Manual}. Addison-Wesley, Reading, MA, USA, 1985.
\bibitem{bib4}
Smith, I.N., R.S. Johnes, and W.P. Hines: 1992, `Title of the Article',
\textit{Journal Title in Italics} \textbf{Vol. no. X}, pp. 00--00
\end{thebibliography}
\end{verbatim}


\begin{figure}[h]
\hspace*{-2.5cm} 
\centering
\includegraphics[width=20.5cm]{cpu-gpu-speedup-CSR}
\caption{Speedup tra CPU e GPU del calcolo con formato CSR}
\end{figure}

\begin{figure}[h]
\hspace*{-2.5cm} 
\centering
\includegraphics[width=20.5cm]{cpu-gpu-speedup-ELLPACK}
\caption{Speedup tra CPU e GPU del calcolo con formato ELLPACK}
\end{figure}

\begin{figure}[h]
\hspace*{-2.5cm} 
\centering
\includegraphics[width=20.5cm]{CPU-CSR-speedup}
\caption{Speedup calcolo su CPU con formato CSR}
\end{figure}

\begin{figure}[h]
\hspace*{-2.5cm} 
\centering
\includegraphics[width=20.5cm]{CPU-ELLPACK-speedup}
\caption{Speedup calcolo su CPU con formato ELLPACK}
\end{figure}

\begin{figure}[h]
\hspace*{-2.5cm} 
\centering
\includegraphics[width=20.5cm]{GPU-CSR-speedup}
\caption{Speedup calcolo su GPU con formato CSR}
\end{figure}

\begin{figure}[h]
\hspace*{-2.5cm} 
\centering
\includegraphics[width=20.5cm]{GPU-ELLPACK-speedup}
\caption{Speedup calcolo su GPU con formato ELLPACK}
\end{figure}

\begin{figure}[h]
\hspace*{-2.5cm} 
\centering
\includegraphics[width=20.5cm]{CPU-CSR-threads}
\caption{GFLOPS al variare dei threads nel calcolo su CPU con formato CSR}
\end{figure}

\begin{figure}[h]
\hspace*{-2.5cm} 
\centering
\includegraphics[width=20.5cm]{CPU-ELLPACK-threads}
\caption{GFLOPS al variare dei threads nel calcolo su CPU con formato ELLPACK}
\end{figure}

\begin{figure}[h]
\hspace*{-2.5cm} 
\centering
\includegraphics[width=20.5cm]{GPU-CSR-blocksize}
\caption{GFLOPS al variare della dimensione del blocco nel calcolo su GPU con formato CSR}
\end{figure}

\begin{figure}[h]
\hspace*{-2.5cm} 
\centering
\includegraphics[width=20.5cm]{GPU-ELLPACK-blocksize}
\caption{GFLOPS al variare della dimensione del blocco nel calcolo su GPU con formato ELLPACK}
\end{figure}


\end{document}
